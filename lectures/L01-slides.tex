\input{configuration}

\title{Lecture 1 ---Introduction and Our C Toolkit }

\author{Jeff Zarnett \\ \small \texttt{jzarnett@uwaterloo.ca}}
\institute{Department of Electrical and Computer Engineering \\
	University of Waterloo}
\date{\today}


\begin{document}

\begin{frame}
	\titlepage

\end{frame}

\begin{frame}
	\frametitle{Course Syllabus}

	As our first order of business, let's go over the course syllabus.

\end{frame}

\begin{frame}
	\frametitle{Collaborative Course}

	The source material for the MTE~241 notes and slides is open-sourced via Github.

	If you find an error in the notes/slides, or have an improvement, go to \url{https://github.com/jzarnett/mte241} and open an issue.

	If you know how to use \texttt{git} and \LaTeX, then you can go to the URL and submit a pull request (changes) for me to look at and incorporate!


\end{frame}

\begin{frame}
\frametitle{Computer Structures and Real-Time Systems}

There's two main areas of discussion in the course.

Computer structures: what is a computer made of?

Real-Time (Operating) Systems: what manages the computer? 

\end{frame}

\begin{frame}
\frametitle{Computer Structures}

To execute a program we need:

	\begin{enumerate}
		\item \textbf{Main Memory}
		\item \textbf{System Bus}
		\item \textbf{Processor}
	\end{enumerate}

	Of course, this is the minimal set.

	\begin{center}
		\includegraphics[width=0.5\textwidth]{images/von-neumann}\\
		Image credit: Wikipedia user Kapooht
	\end{center}



\end{frame}

\begin{frame}
\frametitle{Real-Time Systems}

Real-time systems are those with deadlines. 

They are meant to monitor, interact with, control, or respond to the physical environment.

\end{frame}

\begin{frame}
\frametitle{How are they different?}

\begin{itemize}
	\item Safety
	\item Performance
	\item Fault-Tolerance
	\item Robustness
	\item Scalability
	\item Security
\end{itemize}

\end{frame}

\begin{frame}
\frametitle{Examples of Real-Time Systems}

\begin{itemize}
	\item Traffic control of aircraft
	\item Communications
	\item Nuclear power plant control
	\item Patient monitoring
	\item Smart homes
\end{itemize}

\end{frame}

\begin{frame}
\frametitle{Embedded Systems}

Elicia White's definition of an embedded system is:\\
\quad \textit{a computerized system that is purpose-built for its application.}

Embedded systems usually have limitations or constraints.

\end{frame}

\begin{frame}
\frametitle{Embedded Systems Constraints}
\begin{itemize}
	\item Cost
	\item Correctness requirements
	\item Low memory
	\item Code size restrictions
	\item Processor speed
	\item Power consumption
	\item Available hardware
\end{itemize}
\end{frame}

\begin{frame}
\frametitle{We Need a System}

If the system, embedded or otherwise, is going to do more than one thing, we need some sort of management.

\begin{center}
  \includegraphics[width=0.4\textwidth]{images/incharge.png}
\end{center}

Solution: the Operating System!

\end{frame}

\begin{frame}
\frametitle{Introduction to Operating Systems}

\begin{quote}
\textit{Operating systems are those programs that interface the machine with the applications programs.}
~\\~\\
\textit{The main function of these systems is to dynamically allocate the shared system resources to the executing programs.}
\end{quote}

\hfill - What Can Be Automated?: The Computer Science and Engineering Research Study, MIT Press, 1980

\end{frame}


\begin{frame}
\frametitle{Introduction to Operating Systems}

\begin{center}
	\includegraphics[width=.6\textwidth]{images/linux-user.jpg}
\end{center}


\end{frame}


\begin{frame}
\frametitle{Introduction to Operating Systems}

An operating system (OS) sits between the hardware and programs.

It has many goals, that often conflict with one another.

Its job is to make it so other programs can run efficiently.

\end{frame}

\begin{frame}
\frametitle{Structural Diagram of a Modern Computer}

\begin{center}
\includegraphics[width=0.95\textwidth]{images/os-sw-hw.png}
\end{center}

\end{frame}


\begin{frame}
\frametitle{OS: Resource Manager}

The OS is responsible for resource management and allocation.

Resources like CPU time or memory space are limited.

The OS must decide how to allocate \& to keep track of system resources.

In the event of conflicting requests, choose the winner.


\end{frame}

\begin{frame}
\frametitle{OS: Environment Provider}

The OS enables useful programs like Photoshop or Microsoft Word to run. 

The OS is responsible for abstracting away the details of hardware.

This is so program authors do not have to worry about the specifics.

Imagine Hello World had to be written differently for different hardware.


\end{frame}

\begin{frame}
\frametitle{OS: Multitasking}
Multiple programs means some resources are shared.\\
\quad $\rightarrow$ A source of conflicts!

OS creates and enforces the rules so all can get along.

\begin{center}
	\includegraphics[width=0.4\textwidth]{images/sharing-is-caring.jpg}
\end{center}

Sometimes processes want to co-operate and not compete.\\
\quad The OS can help them to do so.


\end{frame}

\begin{frame}
\frametitle{OS: Efficiency}
Another goal may be to use the computer hardware efficiently.

\begin{center}
	\includegraphics[width=0.7\textwidth]{images/supercomputer.jpg}
\end{center}
\hfill Image Credit: Argonne National Laboratory

Any moment when the supercomputer is not doing useful work is a waste.

\end{frame}

\begin{frame}
\frametitle{OS: What is it, really?}

Operating systems tend to be large and do a lot of things. 

We expect now that an OS comes with a web browser, an e-mail client, some method for editing text, et cetera. 

The part of the operating system we will study is the \alert{Kernel}.

The kernel is the ``core''; the portion of the OS that is always present in main memory and the central part that makes it all work.

\end{frame}

\begin{frame}
\frametitle{OS: Evolution}
Operating systems will evolve over time. 

There will be new hardware released, new types of hardware, new services added, and bug fixes. 

Evolution is constrained: a need to maintain compatibility for programs. 

\begin{center}
	\includegraphics[width=0.7\textwidth]{images/linus-angry.jpg}
\end{center}

\end{frame}


\begin{frame}
\frametitle{Example: SimCity and Windows 95}

\begin{center}
	\includegraphics[width=0.85\textwidth]{images/simcity.png}
\end{center}

\end{frame}


\begin{frame}
\frametitle{Time, What is Time?}

\begin{center}
	\includegraphics[width=0.4\textwidth]{images/10thdoctor.jpg}
\end{center}

\textit{People assume that time is a strict progression from cause to effect, but actually from a non-linear, non-subjective viewpoint, it's more like a big ball of wibbly-wobbly, timey-wimey stuff.}


\end{frame}



\begin{frame}
\frametitle{Real-Time vs. Non Real-Time}

Real-time systems are the ones where wall-clock deadlines matter.

Examples: aviation, industrial machinery, video conferencing, satellites...

\end{frame}

\begin{frame}
\frametitle{Real-Time Scheduling}

There are deadlines, and there are consequences for missing deadlines. 

\begin{center}
	\includegraphics[width=\textwidth]{images/ticktock.jpg}
\end{center}

Fast is not as important as predictable.

\end{frame}

\begin{frame}
\frametitle{Hard and Soft Real-Time}

\alert{Hard real-time}: it has a deadline that must be met to prevent an error, prevent some damage to the system, or for the answer to make sense. 

If a task is attempting to calculate the position of an incoming missile, a late answer is no good. 

A \alert{soft real-time} task has a deadline that is not, strictly speaking, mandatory; missing the deadline degrades the quality of the response, but it is not useless.


\end{frame}


\begin{frame}
	\frametitle{Concurrency}
	A program is said to be concurrent if it can support two or more actions in progress at the same time.

	It is parallel if it can have two or more actions executing simultaneously.

	Soon enough we will spend a great deal of time examining the differences between parallelism and concurrency in the program.

\end{frame}

\begin{frame}
	\frametitle{Concurrency}

	It is already the case that many programs you use are to a greater or smaller degree concurrent.

	Depending on your level of programming experience, you may have already written a concurrent program, intentionally or without knowing it.

	We will learn about how to take a program and make it concurrent, as well as how to write it with concurrency in mind from the ground up.

\end{frame}


\begin{frame}
	\frametitle{Concurrency Problems}

	Consider a program that performs a simple calculation given some input.

	If the program has a concurrency problem,  then the answer could be:

	\begin{enumerate}
		\item Consistently the wrong answer every single time
		\item Different on consecutive runs with the same input, or
		\item Correct some of the time but incorrect some of the time.
	\end{enumerate}

	As you can imagine, none of these options are acceptable.

\end{frame}


\begin{frame}
	\frametitle{Not Gonna Lie...}

	This course is going to be hard.

	\begin{center}
		\includegraphics[width=0.5\textwidth]{images/drago.jpg}
	\end{center}

	If your programming skills need work, better to start trying to catch up now.

\end{frame}


\begin{frame}
	\frametitle{Our C Toolkit}

	We will need some introduction to the conventions and tools of C:
	\begin{itemize}
		\item Functions
		\item Header files
		\item Comments
		\item Structures
		\item Type Names
		\item Memory Allocation, Deallocation, and Pointers
		\item Dereferencing, Address-Of, The Arrow
		\item Arrays
		\item Strings
		\item Calling Conventions \& Errno
		\item Printing
		\item Constants
		\item \texttt{main} and its arguments
		\item \texttt{void*}
	\end{itemize}


\end{frame}




\end{document}

